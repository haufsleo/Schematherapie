\section{Das verletzte Kind}
Der \enquote{Verletzte Kind}-Modus in der Schematherapie bezieht sich auf einen emotionalen Zustand oder eine innere Persönlichkeitskomponente, die die schmerzhaften Erfahrungen, Enttäuschungen und Verletzungen aus der Kindheit repräsentiert. Diese unverarbeiteten emotionalen Wunden können aus Vernachlässigung, Missbrauch, Ablehnung, Verlust oder anderen traumatischen Erfahrungen stammen. Der \enquote{Verletzte Kind}-Modus kann auch dann aktiviert werden, wenn grundlegende emotionale Bedürfnisse in der Kindheit nicht angemessen erfüllt wurden.

Menschen im \enquote{Verletzten Kind}-Modus erleben oft starke negative Emotionen wie Angst, Traurigkeit, Einsamkeit, Verlassenheitsgefühle und Wut. Diese Emotionen können auf aktuelle Situationen übertragen werden und zu ungesunden Denk- und Verhaltensmustern führen. Individuen im \enquote{Verletzten Kind}-Modus können sich in der Gegenwart wie das verletzte Kind von damals fühlen und sich entsprechend verhalten.

Einige typische Merkmale des \enquote{Verletzten Kind}-Modus sind:
\begin{itemize}
    \item \emph{Emotionale Überreaktionen:} Menschen im \enquote{Verletzten Kind}-Modus können auf relativ harmlose Ereignisse oder Bemerkungen mit übermäßiger emotionaler Intensität reagieren. 
    \item \emph{Bedürfnis nach Zuwendung:} Diese Personen können ein tiefes Bedürfnis nach Trost, Verständnis und Unterstützung haben, das aus unerfüllten emotionalen Bedürfnissen in der Kindheit resultiert.
    \item \emph{Negative Selbstbewertung:} Individuen im \enquote{Verletzten Kind}-Modus neigen dazu, sich selbst abzuwerten, sich als minderwertig oder ungeliebt zu empfinden und sich für vergangene Fehler oder Enttäuschungen zu schämen.
    \item \emph{Angst vor Verlassenheit:} Aufgrund der Verletzungen in der Kindheit können sie große Angst davor haben, von anderen abgelehnt oder verlassen zu werden.
    \item  \emph{Rückzug oder Abwehrmechanismen:} Menschen im \enquote{Verletzten Kind}-Modus können sich in schwierigen Situationen emotional zurückziehen oder Abwehrmechanismen wie Wut, Aggressivität oder Passivität einsetzen, um sich vor weiteren Verletzungen zu schützen.    
\end{itemize}
Die Arbeit mit dem \enquote{Verletzten Kind}-Modus in der Schematherapie zielt darauf ab, diese schmerzhaften Erinnerungen und Emotionen anzuerkennen, zu verstehen und zu verarbeiten. Therapeuten helfen den Patienten, Mitgefühl für ihr inneres verletztes Kind zu entwickeln und gesündere Bewältigungsstrategien zu erlernen, um in der Gegenwart besser mit ähnlichen emotionalen Herausforderungen umgehen zu können.



