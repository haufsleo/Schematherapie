Psychische Gesundheitsstörungen können erhebliche Auswirkungen auf das individuelle Wohlbefinden haben. Bipolare Störung, Borderline-Persönlichkeitsstörung, Aufmerksamkeitsdefizit-Hyperaktivitätsstörung (ADHS) und Zwangsstörung sind vier unterschiedliche psychische Erkrankungen, die auf den ersten Blick wenig gemeinsam zu haben scheinen. Bei genauerer Betrachtung lassen sich jedoch Gemeinsamkeiten in Bezug auf bestimmte therapeutische Ansätze erkennen.

In dieser Arbeit werden die Gemeinsamkeiten zwischen diesen vier Erkrankungen genauer untersucht und insbesondere die Anwendung der Schematherapie als therapeutische Intervention betrachtet. Schematherapie ist eine vielversprechende Form der Psychotherapie, die darauf abzielt, tiefsitzende, oft unbewusste Denk- und Verhaltensmuster, sogenannte Schemata, zu identifizieren und zu verändern.

Obwohl jede psychische Störung ihre eigenen spezifischen Merkmale und Diagnosekriterien aufweist, teilen sie dennoch einige Gemeinsamkeiten. Diese Gemeinsamkeiten können die Grundlage für die Anwendung der Schematherapie bilden, um die psychische Gesundheit und das Wohlbefinden der Betroffenen zu verbessern.

In den folgenden Abschnitten werden die wichtigsten Merkmale jeder dieser Störungen sowie die spezifischen Wege, wie die Schematherapie als therapeutischer Ansatz bei der Bewältigung dieser Herausforderungen eingesetzt werden kann, genauer untersucht. Es werden die möglichen Vorteile und Herausforderungen der Anwendung der Schematherapie erörtert.