\section{Symptome und Merkmale}

Die Aufmerksamkeitsdefizit-Hyperaktivitätsstörung (ADHS) ist eine neurobiologische Erkrankung, die hauptsächlich in der Kindheit beginnt und oft bis ins Erwachsenenalter fortbesteht. Sie ist gekennzeichnet durch anhaltende Schwierigkeiten in den Bereichen Aufmerksamkeit, Impulskontrolle und Hyperaktivität. ADHS kann das tägliche Leben, die schulische oder berufliche Leistung und die sozialen Beziehungen erheblich beeinträchtigen. Hier sind die Hauptmerkmale und Symptome der ADHS:

\subsection{Unaufmerksamkeit}

\begin{itemize}
\item \textbf{Schwierigkeiten, die Aufmerksamkeit auf Details zu richten}: Menschen mit ADHS können oft Schwierigkeiten haben, bei Aufgaben oder Aktivitäten auf wichtige Details zu achten, was zu Fehlern führen kann.

\item \textbf{Schwierigkeiten, die Aufmerksamkeit aufrechtzuerhalten}: Betroffene können leicht abgelenkt werden und haben Schwierigkeiten, ihre Aufmerksamkeit über längere Zeiträume auf eine bestimmte Aufgabe zu richten.

\item \textbf{Vergesslichkeit}: Häufiges Vergessen von Alltagsaufgaben, Terminen oder Arbeitsmaterialien ist ein häufiges Symptom.

\item \textbf{Organisationsprobleme}: Schwierigkeiten bei der Organisation von Aufgaben und Aktivitäten können zu Chaos und Unordnung führen.
\end{itemize}

\subsection{Hyperaktivität}

\begin{itemize}
\item \textbf{Unruhe und Bewegungsdrang}: Hyperaktive Symptome manifestieren sich durch anhaltenden Bewegungsdrang, Zappeln oder das Gefühl, immer in Bewegung sein zu müssen.

\item \textbf{Schwierigkeiten ruhig zu sitzen}: Personen mit ADHS finden es oft schwer, ruhig zu sitzen, insbesondere in Situationen, die Geduld erfordern, wie in der Schule oder bei Besprechungen.

\item \textbf{Reden ohne Unterbrechung}: Übermäßiges Reden, ohne auf andere zu hören oder sie ausreden zu lassen, ist ein weiteres charakteristisches Merkmal.
\end{itemize}

\subsection{Impulsivität}

\begin{itemize}
\item \textbf{Impulsive Entscheidungen}: Betroffene neigen dazu, spontane Entscheidungen zu treffen, ohne die Konsequenzen zu bedenken, was zu Fehlern oder riskantem Verhalten führen kann.

\item \textbf{Schwierigkeiten, auf den richtigen Moment zu warten}: Geduld und das Abwarten in Wartesituationen sind oft eine Herausforderung.

\item \textbf{Unterbrechung anderer}: Menschen mit ADHS können dazu neigen, andere in Gesprächen oder Aktivitäten zu unterbrechen.
\end{itemize}

\subsection{Dauerhaftkeit und Schweregrad}

Um eine Diagnose von ADHS zu erhalten, müssen die oben genannten Symptome mindestens sechs Monate lang bestehen und in verschiedenen Lebensbereichen wie Schule, Arbeit und sozialen Beziehungen auftreten. Der Schweregrad der Symptome kann von Person zu Person variieren und reicht von milden bis zu schweren Ausprägungen.

Es ist wichtig zu beachten, dass ADHS nicht einfach auf mangelnde Selbstkontrolle oder schlechtes Benehmen zurückzuführen ist. Es handelt sich um eine neurobiologische Störung, bei der bestimmte Gehirnstrukturen und Neurotransmitter beteiligt sind. Die genaue Ursache von ADHS ist noch nicht vollständig verstanden, aber genetische, umweltbedingte und neurobiologische Faktoren spielen eine Rolle.

Die Diagnose von ADHS erfordert eine umfassende Untersuchung durch qualifizierte Fachleute, darunter Ärzte, Psychiater und Psychologen. Die Behandlung kann eine Kombination aus Verhaltenstherapie, Medikamenten und Unterstützung im Bildungsbereich umfassen. Mit einer geeigneten Behandlung und Unterstützung können Menschen mit ADHS erfolgreich Strategien entwickeln, um ihre Symptome zu bewältigen und ein erfülltes Leben zu führen. Frühzeitige Erkennung und Intervention sind entscheidend, um die Lebensqualität von Betroffenen zu verbessern.

\section{Schematherapie bei ADHS}

Die Aufmerksamkeitsdefizit-Hyperaktivitätsstörung (ADHS) ist eine neurobiologische Erkrankung, die sich durch anhaltende Probleme mit Aufmerksamkeit, Impulskontrolle und Hyperaktivität auszeichnet. Während die Behandlung von ADHS häufig Medikamente und Verhaltenstherapie umfasst, gewinnt auch die Schematherapie zunehmend an Bedeutung als ergänzender Ansatz zur Bewältigung der Herausforderungen, die mit dieser Störung einhergehen.

\subsection{Was ist Schematherapie?}

Die Schematherapie ist eine Form der Psychotherapie, die darauf abzielt, tiefsitzende, oft unbewusste Denk- und Verhaltensmuster, sogenannte Schemata, zu identifizieren und zu verändern. Diese Schemata können aus negativen Erfahrungen in der Kindheit resultieren und zu emotionalen Problemen und zwischenmenschlichen Schwierigkeiten im Erwachsenenalter führen. Die Schematherapie zielt darauf ab, diese Schemata zu erkennen, zu verstehen und durch gesündere Alternativen zu ersetzen.

\subsection{Wie kann Schematherapie bei ADHS helfen?}

Die Schematherapie kann bei ADHS auf verschiedene Weisen hilfreich sein:

\subsubsection{Selbstverständnis und Akzeptanz}

Menschen mit ADHS neigen dazu, sich selbst oft kritisch zu sehen und sich für ihre Schwierigkeiten zu verurteilen. Die Schematherapie kann dazu beitragen, ein tieferes Verständnis für die zugrunde liegenden emotionalen Bedürfnisse und Schemata zu entwickeln, die dazu führen, dass sie sich anders verhalten als Menschen ohne ADHS. Dies kann zu einer besseren Selbstakzeptanz führen und die Scham reduzieren, die oft mit ADHS verbunden ist.

\subsubsection{Emotionsregulation}

Eine der Herausforderungen bei ADHS ist die Schwierigkeit, Emotionen zu regulieren. Menschen mit ADHS können dazu neigen, impulsiver auf emotional belastende Situationen zu reagieren. Die Schematherapie kann Techniken zur Emotionsregulation vermitteln, um den Umgang mit intensiven Emotionen zu verbessern und impulsives Verhalten zu reduzieren.

\subsubsection{Bewältigungsstrategien}

Die Schematherapie kann konkrete Bewältigungsstrategien vermitteln, die auf die individuellen Bedürfnisse und Schwierigkeiten eines Menschen mit ADHS zugeschnitten sind. Dies kann den Umgang mit Prokrastination, Zeitmanagement, Organisationsproblemen und anderen Herausforderungen erleichtern.

\subsubsection{Verbesserung der zwischenmenschlichen Beziehungen}

ADHS kann zwischenmenschliche Beziehungen belasten, da Menschen mit ADHS oft als unaufmerksam, impulsiv oder unzuverlässig wahrgenommen werden. Die Schematherapie kann dazu beitragen, die Kommunikationsfähigkeiten und sozialen Fertigkeiten zu verbessern, um bessere Beziehungen aufzubauen und aufrechtzuerhalten.

\subsubsection{Förderung der Selbstkontrolle}

Die Schematherapie kann Techniken zur Selbstkontrolle und Impulskontrolle vermitteln, um impulsives Verhalten zu reduzieren und die Fähigkeit zur langfristigen Zielverfolgung zu verbessern.

\subsection{Die Rolle eines qualifizierten Therapeuten}

Es ist wichtig zu betonen, dass die Schematherapie von einem qualifizierten Therapeuten durchgeführt werden sollte, der Erfahrung in der Arbeit mit Menschen mit ADHS hat. Die Therapie sollte individuell auf die Bedürfnisse und Ziele des Patienten zugeschnitten sein.

Die Schematherapie allein ersetzt in der Regel nicht die medikamentöse Behandlung oder andere therapeutische Ansätze bei ADHS. Die Schematherapie kann als ergänzende Therapieoption dienen, um die Bewältigung der Herausforderungen im Alltag bei ADHS zu unterstützen.

Insgesamt stellt die Schematherapie für Menschen mit ADHS eine wertvolle Ressource dar, um ihr Selbstverständnis, ihre Emotionsregulation, ihre sozialen Fertigkeiten und ihre Selbstkontrolle zu verbessern. Durch die Arbeit an den zugrunde liegenden Schemata können Menschen mit ADHS ihre
Lebensqualität erhöhen und effektiver mit den Herausforderungen ihrer Erkrankung umgehen. Es ist jedoch wichtig zu betonen, dass die Schematherapie nicht für jeden gleich wirksam ist. Bei der Auswahl einer geeigneten Therapieoption sollten die individuellen Bedürfnisse und Ziele berücksichtigt werden.

Zusammenfassend kann die Schematherapie bei ADHS helfen, indem sie das Selbstverständnis fördert, die Emotionsregulation verbessert, Bewältigungsstrategien vermittelt, zwischenmenschliche Beziehungen stärkt und die Selbstkontrolle fördert. Eine qualifizierte Therapie kann die Wirksamkeit dieses Ansatzes maximieren und dazu beitragen, dass Menschen mit ADHS ein erfüllteres und erfolgreicheres Leben führen können. Wenn Sie oder jemand, den Sie kennen, von ADHS betroffen ist, ist es ratsam, professionelle Hilfe in Anspruch zu nehmen, um die besten individuellen Behandlungsoptionen zu ermitteln und die Lebensqualität zu verbessern.

\section{ADHS und das Innere Kind}
%
Das Konzept des \enquote{Inneren Kindes} aus der Schematherapie bezieht sich auf die emotionalen Erfahrungen und Bedürfnisse aus der Kindheit, die im Erwachsenenalter immer noch Auswirkungen haben können. Bei Menschen mit ADHS (Aufmerksamkeitsdefizit-/Hyperaktivitätsstörung) kann das innere Kind auf verschiedene Arten und in unterschiedlichen Situationen zum Ausdruck kommen:
\begin{itemize}
    \item \textbf{Impulsives Verhalten:} Menschen mit ADHS könnten in manchen Situationen impulsiver handeln, ohne vollständig über die Konsequenzen nachzudenken. Dieses impulsive Verhalten kann Ähnlichkeiten mit kindlichem Verhalten aufweisen.
    \item \textbf{Kreativität und Neugier:} Das innere Kind kann sich in der kreativen und neugierigen Seite von Menschen mit ADHS zeigen. Sie könnten eine lebhafte Vorstellungskraft haben und Interesse an verschiedenen Aktivitäten und Ideen zeigen.
    \item \textbf{Ungeduld und Frustration:} Wenn die Dinge nicht so laufen, wie sie es sich wünschen, könnten Menschen mit ADHS eine erhöhte Frustration oder Ungeduld zeigen, die auf unerfüllte emotionale Bedürfnisse oder eine niedrige Frustrationstoleranz aus der Kindheit zurückzuführen sein könnten.
    \item \textbf{Schwierigkeiten mit Struktur:} Das innere Kind könnte sich in Schwierigkeiten mit der Organisation und Strukturierung des Alltags zeigen. Menschen mit ADHS könnten Mühe haben, Aufgaben zu planen und zu organisieren, ähnlich wie es bei Kindern oft der Fall ist.
    \item \textbf{Spontaneität:} Das innere Kind kann sich in der Spontaneität und dem Bedürfnis nach Abwechslung zeigen. Menschen mit ADHS könnten Schwierigkeiten haben, sich über längere Zeit auf eine Sache zu konzentrieren, und stattdessen nach neuen und aufregenden Aktivitäten suchen.
    \item \textbf{Emotionale Intensität:} Menschen mit ADHS könnten in ihren Emotionen intensiver sein und starke Gefühle von Freude, Frustration oder Überforderung erleben, ähnlich wie es bei Kindern der Fall ist.
    \item \textbf{Gefühl der Andersartigkeit:} Das innere Kind könnte in einem Gefühl der Andersartigkeit oder des Nicht-Verstanden-Werdens Ausdruck finden, das auf Erfahrungen in der Kindheit zurückgeführt werden kann.
\end{itemize}
%
Es ist wichtig zu betonen, dass das innere Kind bei Menschen mit ADHS in verschiedenen Facetten auftreten kann. Dieses Konzept kann dazu beitragen, eine tiefere Verbindung zu den emotionalen Aspekten der Störung herzustellen und den persönlichen Heilungsprozess zu unterstützen. Ein professioneller Therapeut kann helfen, die Verbindung zwischen ADHS-Symptomen und emotionalen Bedürfnissen aus der Kindheit zu verstehen und zu bearbeiten.



