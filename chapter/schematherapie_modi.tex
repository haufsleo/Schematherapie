\section{Welche Modi gibt es?}
In der Schematherapie gibt es verschiedene Modi, die emotionale Zustände oder Persönlichkeitsaspekte repräsentieren. Jeder Modus ist mit bestimmten Denk- und Verhaltensmustern verbunden, die in bestimmten Situationen auftreten können. Hier sind einige der häufig vorkommenden Modi:
\begin{itemize}
    \item \emph{Das verletzte Kind:} Dieser Modus repräsentiert die schmerzhaften Erfahrungen und Emotionen aus der Kindheit, die nicht verarbeitet wurden. Menschen im verletzten Kind-Modus können sich hilflos, ängstlich, traurig oder verärgert fühlen. Sie neigen dazu, sich zurückgezogen zu verhalten oder emotional überreagieren.
    \item \emph{Das innere Kind:} Dieser Modus bezieht sich auf die positiven und lebensfrohen Aspekte der Kindheit. Menschen im inneren Kind-Modus können spielerisch, kreativ und neugierig sein. Dieser Modus kann aber auch zu impulsivem Verhalten und mangelnder Selbstkontrolle führen. 
    \item \emph{Der strenge Elternteil:} Dieser Modus repräsentiert kritische und strenge innere Stimmen, die von kritischen oder überfordernden Bezugspersonen in der Kindheit stammen können. Menschen im strengen Eltern-Modus können sich selbst hart beurteilen, sich schuldig fühlen und sich hohe Erwartungen setzen.
    \item \emph{Der fürsorgliche Elternteil:} Dieser Modus enthält die liebevollen, unterstützenden und beruhigenden Aspekte der inneren Stimmen. Menschen im fürsorglichen Eltern-Modus können sich selbst mitfühlend behandeln und sich selbst Trost spenden.
    \item \emph{Der erwachsene Modus:} Dieser Modus repräsentiert die erwachsene, rationale und realistische Seite einer Person. Im erwachsenen Modus können Menschen rationale Entscheidungen treffen und konstruktive Handlungen ausführen.
    \item \emph{Der abtrünnige Modus:} Dieser Modus enthält aggressive oder auch selbst zerstörerische Tendenzen. Menschen im abtrünnigen Modus können impulsiv, wütend oder sogar destruktiv handeln und sich selbst oder anderen Schaden zufügen.
    \item \emph{Der glückliche Modus:} Dieser Modus repräsentiert Freude, Genuss und Entspannung. Menschen im glücklichen Modus können positive Emotionen erleben und sich erholen.
\end{itemize}
Diese Modi sind nicht statisch und können sich je nach Situation oder Stressfaktoren verändern. In der Schematherapie ist es wichtig, diese verschiedenen Modi zu erkennen, zu verstehen und zu lernen, mit ihnen auf gesunde Weise umzugehen, um eine verbesserte emotionale Regulation und Verhaltensänderung zu erreichen.
