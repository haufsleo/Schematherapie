\section{Was ist Schematherapie}
Die Schematherapie ist eine psychotherapeutische Methode, die Menschen dabei hilft, tief verwurzelte dysfunktionale Denk- und Verhaltensmuster, sogenannte \enquote{Schemata}, zu identifizieren und zu bewältigen. Diese Schemata entstehen oft in der Kindheit aufgrund von traumatischen Erfahrungen, negativen Beziehungsmustern oder unzureichender emotionaler Versorgung. Über längere Zeit können emotionale Schwierigkeiten, Beziehungsprobleme und psychische Störungen entstehen.

Die Schematherapie vereint Elemente aus verschiedenen psychotherapeutischen Ansätzen wie der kognitiven Verhaltenstherapie, der Gestalttherapie, der psychodynamischen Therapie und der emotionalen Fokustherapie. Der Therapieansatz zielt darauf ab, die zugrunde liegenden Schemata zu erkennen, die zur Entstehung von ungesunden Verhaltens- und Denkmustern führen, und sie durch adaptive Alternativen zu ersetzen.

Ein zentrales Element der Schematherapie ist die Arbeit mit sogenannten Modi, das sind emotionale Zustände oder Teile der Persönlichkeit, die in bestimmten Situationen aktiviert werden. Diese Modi können aus den ursprünglichen Schemata abgeleitet werden und verstärken negative Emotionen sowie destruktive Verhaltensweisen. Die Therapie hilft den Patienten dabei, diese Modi zu erkennen, zu verstehen und mit ihnen umzugehen, um gesündere Bewältigungsstrategien zu entwickeln.

Die Schematherapie umfasst oft sowohl individuelle als auch gruppenbasierte Ansätze. Sie kann bei einer Vielzahl von psychischen Störungen und emotionalen Problemen eingesetzt werden, einschließlich Persönlichkeitsstörungen, Angststörungen, Depressionen und Beziehungsproblemen. Die Therapie erfordert in der Regel eine aktive Zusammenarbeit zwischen dem Therapeuten und dem Patienten, um langfristige Veränderungen in Denk- und Verhaltensmustern zu erreichen.