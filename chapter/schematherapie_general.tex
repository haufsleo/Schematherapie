\section{Was ist Schematherapie}
Die Schematherapie ist eine psychotherapeutische Methode, die Menschen hilft, tief verwurzelte dysfunktionale Denk- und Verhaltensmuster, so genannte \enquote{Schemata}, zu erkennen und zu überwinden. Diese Muster entstehen oft in der Kindheit aufgrund traumatischer Erfahrungen, negativer Beziehungsmuster oder unzureichender emotionaler Zuwendung. Im Laufe der Zeit können sich daraus emotionale Schwierigkeiten, Beziehungsprobleme und psychische Störungen entwickeln.

Die Schematherapie verbindet Elemente aus verschiedenen psychotherapeutischen Ansätzen wie der kognitiven Verhaltenstherapie, der Gestalttherapie, der psychodynamischen Therapie und der emotionsfokussierten Therapie. Der therapeutische Ansatz zielt darauf ab, die zugrunde liegenden Schemata, die zur Entstehung ungesunder Verhaltens- und Denkmuster führen, zu erkennen und durch passende Alternativen zu ersetzen.

Ein zentrales Element der Schematherapie ist die Arbeit mit so genannten Modi, also emotionalen Zuständen oder Persönlichkeitsanteilen, die in bestimmten Situationen aktiviert werden. Diese Modi lassen sich aus den ursprünglichen Schemata ableiten und verstärken negative Emotionen und destruktives Verhalten. Die Therapie hilft den Patienten, diese Modi zu erkennen, zu verstehen und mit ihnen umzugehen, um gesündere Bewältigungsstrategien zu entwickeln.

Die Schematherapie umfasst häufig sowohl Einzel- als auch Gruppenansätze. Sie kann bei einer Vielzahl von psychischen Störungen und emotionalen Problemen eingesetzt werden, einschließlich Persönlichkeitsstörungen, Angststörungen, Depressionen und Beziehungsproblemen. Die Therapie erfordert in der Regel eine aktive Zusammenarbeit zwischen Therapeut und Patient, um langfristige Veränderungen in Denk- und Verhaltensmustern zu erreichen.