\section{Was ist Schematherapie}
Die Schematherapie ist eine psychotherapeutische Methode, die entwickelt wurde, um Menschen dabei zu helfen, tief verwurzelte dysfunktionale Denk- und Verhaltensmuster, die als \enquote{Schemata} bezeichnet werden, zu identifizieren und zu bewältigen. Diese Schemata entstehen oft in der Kindheit aufgrund von traumatischen Erfahrungen, negativen Beziehungsmustern oder unzureichender emotionaler Versorgung. Sie können im Laufe der Zeit zu anhaltenden emotionalen Schwierigkeiten, Beziehungsproblemen und psychischen Störungen führen.

Die Schematherapie integriert Elemente aus verschiedenen psychotherapeutischen Ansätzen, darunter kognitive Verhaltenstherapie, Gestalttherapie, psychodynamische Therapie und emotionale Fokustherapie. Der Therapieansatz zielt darauf ab, die zugrunde liegenden Schemata zu erkennen, die zur Entstehung von ungesunden Verhaltens- und Denkmustern führen, und sie durch adaptive Alternativen zu ersetzen.

Ein zentrales Element der Schematherapie ist die Arbeit mit \enquote{Modi}, das sind emotionale Zustände oder Teile der Persönlichkeit, die in bestimmten Situationen aktiviert werden. Diese Modi können sich aus den ursprünglichen Schemata ableiten und negative Emotionen sowie destruktive Verhaltensweisen verstärken. Die Therapie hilft den Patienten, diese Modi zu erkennen, zu verstehen und mit ihnen umzugehen, um gesündere Bewältigungsstrategien zu entwickeln.

Die Schematherapie umfasst oft sowohl individuelle als auch gruppenbasierte Ansätze. Sie kann bei einer Vielzahl von psychischen Störungen und emotionalen Problemen eingesetzt werden, einschließlich Persönlichkeitsstörungen, Angststörungen, Depressionen und Beziehungsproblemen. Die Therapie erfordert in der Regel eine aktive Zusammenarbeit zwischen dem Therapeuten und dem Patienten, um langfristige Veränderungen in Denk- und Verhaltensmustern zu erreichen.