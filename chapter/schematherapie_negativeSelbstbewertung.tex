\section{Negative Selbstbewertung}

Negative Selbstbewertung, auch als negative Selbstwahrnehmung oder Selbstabwertung bezeichnet, ist ein psychologisches Konzept, das sich auf die Neigung einer Person bezieht, sich selbst in negativen und herabsetzenden Begriffen zu sehen. Diese Selbstbewertung kann das Selbstwertgefühl, die Selbstachtung, die emotionalen Reaktionen und das Verhalten einer Person beeinträchtigen.

Personen mit einer ausgeprägten negativen Selbstbewertung tendieren dazu, sich selbst aufgrund ihrer Fehler, vermeintlichen Mängel oder negativen Eigenschaften zu kritisieren und zu verurteilen. Sie neigen dazu, ihre eigenen positiven Qualitäten und Erfolge herunterzuspielen oder zu ignorieren und sich stattdessen auf ihre vermeintlichen Schwächen zu konzentrieren. Diese negative Selbstwahrnehmung kann sich auf verschiedene Lebensbereiche auswirken, wie zum Beispiel Beziehungen, Beruf und persönliche Erfüllung.

Einige Merkmale und Auswirkungen der negativen Selbstbewertung sind:
\begin{itemize}
    \item \textbf{Selbstkritik:} Menschen mit negativer Selbstbewertung sind oft sehr selbstkritisch und neigen dazu, sich für Fehler oder Unzulänglichkeiten hart zu beurteilen.
    \item \textbf{Geringes Selbstwertgefühl:} Diese Personen haben oft ein niedriges Selbstwertgefühl und fühlen sich minderwertig im Vergleich zu anderen.
    \item \textbf{Perfektionismus:} Negative Selbstbewertung kann zu Perfektionismus führen, da die Person ständig danach strebt, sich zu verbessern, um ihre eigenen hohen Standards zu erfüllen.
    \item \textbf{Vermeidungsverhalten:} Menschen mit negativer Selbstbewertung könnten sich aus Angst vor Versagen oder Ablehnung bestimmten Aktivitäten oder Situationen entziehen.
    \item \textbf{Depression und Angst:} Eine anhaltende negative Selbstbewertung kann das Risiko für Depressionen, Angststörungen und andere psychische Gesundheitsprobleme erhöhen.
    \item \textbf{Beziehungsprobleme:} Negative Selbstbewertung kann dazu führen, dass Menschen Schwierigkeiten haben, sich in Beziehungen wohl zu fühlen, da sie befürchten, nicht wertvoll genug zu sein oder abgelehnt zu werden.
    \item \textbf{Selbsterfüllende Prophezeiung:} Die negative Selbstbewertung kann dazu führen, dass Menschen sich selbst in Situationen hineinführen, in denen sie tatsächlich weniger erfolgreich sind, da sie von vornherein glauben, dass sie scheitern werden.
\end{itemize}
%
Therapieansätze wie die kognitive Verhaltenstherapie und Schematherapie können Menschen dabei unterstützen, ihre negative Selbstbewertung zu erkennen, zu hinterfragen und zu verändern. Dabei lernen sie, realistischere und positivere Sichtweisen über sich selbst zu entwickeln und gesündere Bewältigungsstrategien zu erlernen, um mit Selbstkritik und negativen Gedanken umzugehen.


